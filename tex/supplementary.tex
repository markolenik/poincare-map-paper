\documentclass[utf8]{frontiers_suppmat} % for all articles
\usepackage{url,hyperref,lineno,microtype}
\usepackage[onehalfspacing]{setspace}
\usepackage{cleveref}
\usepackage{physics}
\usepackage{siunitx}
\usepackage{xr}
\externaldocument{draft}


\renewcommand{\d}{\mathrm{d}}
\renewcommand{\k}{\mathrm{K}}
\newcommand{\ca}{\mathrm{Ca}}
\newcommand{\na}{\mathrm{Na}}
\newcommand{\leak}{\mathrm{L}}
\newcommand{\dstar}{d^\star}
\newcommand{\gstar}{g^\star}
\newcommand{\gbar}{\bar g}
\newcommand{\delt}{\Delta t}
\newcommand{\taus}{\tau_s}
\newcommand{\dn}{\delta_n}
\newcommand{\taud}{\tau_d}


% Leave a blank line between paragraphs instead of using \\

\begin{document}
\onecolumn
\firstpage{1}

\title[Supplementary Material]{{\helveticaitalic{Supplementary Material}}}


\maketitle

\section{Model equations and parameters}
The asymptotic function for the calcium conductance $m_{\infty}$ is given by
\begin{equation}
	m_{\infty}(v) = \frac{1}{2}\left(1+\tanh{\left((v-v_{A})/v_{B}\right)}\right).
\end{equation}
The voltage and potassium nullclines, \(v_{\infty}(v)\) and \(w_{\infty}(v)\), are
\begin{align}
	v_{\infty}(v) & = \frac{-g_{l}(v-v_{l}) - g_{ca}m_{\infty}(v)(v-v_{ca}) + I - \bar g s(v-v_{s})}{g_{k}(v-v_{k})},
	\\
	w_{\infty}(v) & = \frac{1}{2}\left(1+\tanh{\left((v-v_{C})/v_{D}\right)}\right).
\end{align}
Model parameters were adapted from~\cite{bose2011} and are given in \cref{tab:pars}:
\begin{table}[h]
	\caption{Default parameters for coupled Morris-Lecar model.~\label{tab:pars}}
	\centering
	\begin{tabular}{ll}
		Parameter                        & value                 \\
		\hline
		                                 &                       \\
		$g_{\leak}$                      & $0.15$ \si{mS/cm^2}   \\
		$g_{\ca}$                        & $0.3$ \si{mS/cm^2}    \\
		$g_{\k}$                         & $0.6$ \si{mS/cm^2}    \\
		$v_{\leak}$                      & $-50$ \si{mV}         \\
		$v_{\ca}$                        & $100$ \si{mV}         \\
		$v_{\k}$                         & $-70$ \si{mV}         \\
		$v_{A}$                          & $1$ \si{mV}           \\
		$v_{B}$                          & $14.5$ \si{mV}        \\
		$v_{C}$                          & $4$ \si{mV}           \\
		$v_{D}$                          & $15$ \si{mV}          \\
		$I$                              & $3.8$ \si{\mu A/cm^2} \\
		$\tau_w$                         & $100$ \si{ms}         \\
		$\tau_a$                         & $1000$ \si{ms}        \\
		$\tau_b$                         & $100$ \si{ms}         \\
		$\tau_\kappa$                    & $100$ \si{ms}         \\
		$\tau_\gamma$                    & $0.0001$ \si{ms}      \\
		$v_{\theta}$                     & $0$ \si{mV}           \\
		$v_{s}$                          & $-80$ \si{mV}         \\
		$T$                              & $376$ \si{ms}         \\
		$T_{act}$                        & $49$ \si{ms}          \\
		$T_{inact}$                      & $327$ \si{ms}         \\
		$\gstar$                         & $0.0036$ \si{mS/cm^2} \\
		$\lambda:=\exp(-T_{act}/\tau_b)$ & $0.612$               \\
		$\rho:=\exp(-T_{inact}/\tau_a)$  & $0.721$               \\
	\end{tabular}
\end{table}


\section{Computing bifurcation diagram numerically}\label{appendix2}
The bifurcation diagram of stable $n-n$ solutions of the two-cell network in \cref{fig:bif-diagram} is obtained numerically as follows:
We initialise the coupling strength at parameter values associated with one type of $n-n$ solution, that is we choose the values $\bar g = 0.4, 0.7, 0.8, 0.9, 0.98$ for the $1-1$, $2-2$, $3-3$, $4-4$, and $5-5$ solutions respectively.
For each $\bar g$ the system is then numerically integrated sufficiently long for any transients to fully subside.
We then identify one period of the solution by finding the first return of the depression variable $d_{1}$.
That is, we choose some value $d_{k}$ at a spike time $t_{k}$, and by iterating from spike to spike find some subsequent value $d_{k+1}$ at spike time such that $|d_{k+1}-d_{k}|<\epsilon$.
If a periodic solution of type $n-n$ is found in such way, $\bar g$ is step-wise increased/decreased, and the above algorithm is repeated.
Otherwise, the set of all previously found solutions and the corresponding values $\bar g$ are returned.

Note that applying conventional methods of numerical continuation is not straightforward for our model, as the discontinuous resets in \cref{eq:s-reset} and~\eqref{eq:d-reset} make the root finding challenging~\citep[e.g. see][ for continuation methods]{kuznetsov2004}.



\bibliographystyle{frontiersinHLTH&FPHY} % for Health, Physics and Mathematics articles
\bibliography{bibliography.bib}

\end{document}

%%% Local Variables:
%%% mode: latex
%%% TeX-master: t
%%% End:
