\begin{abstract}
  \section{}
  Short-term synaptic plasticity is found in many areas of the central nervous system.
  In the inhibitory half-centre central pattern generators involved in locomotion, synaptic depression is believed to act as a burst termination mechanism, allowing networks to generate anti-phase bursting patterns of varying periods.
  To better understand burst generation in these central patter generators, we study a minimal network of two neurons coupled through depressing synapses.
  Depending on the strength of the synaptic conductance between the two neurons, this network can produce symmetric $n:n$ anti-phase bursts, where neurons fire $n$ spikes in alternation, with the period of such solutions increasing with the strength of the synaptic conductance.
  Relying on the timescale disparity in the model, we reduce the eight-dimensional network equations to a fully-explicit scalar Poincaré burst map.
  This map tracks the state of synaptic depression from one burst to the next and captures the complex bursting dynamics of the network.
  Fixed points of this map are associated with stable burst solutions of the full network model, and are created through fold bifurcations of maps.
  \linelabel{line:abstract}
  We derive conditions that predict the bifurcations between $n:n$ and $(n+1):(n+1)$ solutions, producing a full bifurcation diagram of the burst cycle period.
  Predictions of the Poincaré map fit excellently with numerical simulations of the full network model and allow the study of parameter sensitivity for rhythm generation.


  \tiny
  \keyFont{ \section{Keywords:} Synaptic depression, Poincaré map, Dynamical system, Neuronal bursting, Central pattern generator}

\end{abstract}


\section{Introduction}
% DEPRESSION
Short-term synaptic plasticity may have a role in burst activity in central pattern
generators (CPGs).
Short-term synaptic depression is commonly found in neuronal networks involved in the generation of rhythmic movements, such as in the pyloric CPG of the spiny lobster~\citep{manor1997, rabbah2007}, or in the lumbosacral cord of the chick embryo~\citep{donovan1998}.
Synaptic depression modulates the strength of synapses in response to changes to the presynaptic firing frequency.
At a high neuronal firing frequency, depression weakens the strength of synapses and therefore reduces the magnitude of the postsynaptic response.
At low firing frequency, it allows sufficient time for the synapse to recover from depression between spikes, leading to a stronger postsynaptic response.
In reciprocal networks, synaptic depression has been shown to act as a ``switch'', giving rise to a wide range of network dynamics such as synchronous and multi-stable rhythms, as well as fine tuning the frequency of network oscillations~\citep{nadim2000, nadim1999, bose2011}.

% Brown
\linelabel{line:A}
~\citet{brown1911} pioneered the idea that synaptic depression acts as a burst termination
mechanism in CPGs composed of reciprocally inhibitory neurons and involved in rhythm generation of locomotion.
When one side is firing during a burst the other, antagonistic side, is prevented from
firing by synaptic inhibition.
However, the weakening of inhibition as a result of synaptic depression eventually releases the antagonistic side so that it starts firing, terminating the burst on the side that had originally been firing.
This rhythmogenesis hypothesis has been considered one of a handful of standard mechanisms for generating locomotion rhythms in vertebrates~\citep{reiss1962,perkel1974,friesen1994}.
It has been proposed as an explanation of the antiphase burst rhythm in struggling in \textit{Xenopus} tadpoles~\citep{li2007}.

~\citet{bose2011} investigated burst generation in a generic half-centre CPG that consists
of two identical, tonically active Morris-Lecar~\citep{morris1981} neurons coupled through
inhibitory depressing synapses.
Numerical simulations showed that when the reciprocal synaptic conductance between the two neurons is varied, the network produces symmetric $n:n$ anti-phase bursts, with stronger synaptic coupling leading to longer bursts.
They used methods from geometric singular perturbation theory to separate the timescales of the fast membrane, and the slow synaptic dynamics of the network to derive one-dimensional conditions necessary for the existence of stable $n:n$ solutions (for $n\leq 2$).
According to these conditions the type of firing pattern largely depends on the slow
depression dynamics of the synapses between the two neurons, and can therefore be
predicted by knowing the strengths of the synaptic conductances of the two synapses.
\linelabel{line:B}
Thus, the scalar conditions derived in \citet{bose2011} provide a method to numerically
identify the type of stable $n:n$ pattern for any given value of the coupling strength and $n\leq 2$.
However, they do not predict the exact period of such solutions.
Furthermore, while they provide good arguments for the validity of their reduction assumptions and the resulting scalar conditions, they do not verify them numerically.

Here we extend the previous analysis by providing a Poincaré map of the slow depression
dynamics.
This allows us not only to predict the types of stable $n:n$ solutions the full
network can produce, (for any $n$), but also to study how varying the coupling strength
affects the period of such solutions.
To do this, we build on, and numerically test, the
assumptions on the fast-slow timescale disparity made in~\cite{bose2011}.
We reduce the two-cell model to a scalar Poincaré map that tracks the evolution of the depression from the beginning of one burst to the beginning of the next burst.
Stable fixed points of our map are associated with stable $n:n$ burst solutions.
Our map construction is motivated by the burst length map of a T-type calcium current, utilised by~\citet{matveev2007}, which approximates the anti-phase bursting dynamics of a network of two coupled Morris-Lecar neurons.
In contrast to our model, the network described in the~\cite{matveev2007} paper does not contain short-term synaptic depression, and burst termination is instead accomplished through the dynamics of a slow T-type calcium current.

% MAIN RESULTS AND CONCLUSIONS
The Poincaré map derived here replicates the results from numerical simulations of the
full two-cell ODE system: Given the strength of maximum conductance between the two
neurons, fixed points of our map predict the type and period of $n:n$ patterns, the switch
between burst solutions of different periods, as well as the occurrence of co-existent
solutions.
In addition to proving the existence and stability of fixed points, our map shows that fixed points are created via a fold bifurcation of maps.
Finally, we use our map to derive algebraic conditions that allow us to predict parameter values of the maximum conductance at which $n:n$ solutions bifurcate to $(n+1):(n+1)$ solutions, and vice versa.
Because our map is fully explicit, it lays the framework for studying the effects of other model parameter on network dynamics without the need to run expensive numerical integrations of the ODEs.

% PAPER STRUCTURE.
This paper is organised as follows.
First, we introduce the network of two neurons, and describe the properties of single cell and synapse dynamics.
We use numerical simulations of the network to provide an intuition for the range of possible burst dynamics the system can produce.
Next, we state and justify the simplifying assumptions that are necessary for the map construction.
Finally, we analytically derive the first return map of the depression variable as well as the conditions that are required for stable $n:n$ solutions.
We end this work with a discussion.
