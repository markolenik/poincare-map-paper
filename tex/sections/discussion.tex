\section{Discussion}
% PROBLEM SUMMARY
Synaptic depression of inhibition is believed to play an important role in the generation of rhythmic activity involved in many motor rhythms such as in leech swimming \citep{mangan1994} and leech heart beat~\citep{calabrese1995}, and in the lobster pyloric system~\citep{manor1997, rabbah2007}.
In inhibitory half-centre CPGs, such as believed to be found in the struggling network of \textit{Xenopus} tapdoles,  synaptic depression can act as a burst termination mechanism, enabling the alternation of bursting between the two sides of the CPG~\citep{li2007}.
\linelabel{line:principle}
Modelling can shed light on the underlying mathematical principles that enable the generation of such anti-phase bursts, and help identify the components that control this rhythm allowing it to switch between different patterns.

% MAIN RESULTS
To study the mechanisms of burst generation in half-centre CPGs we have analysed a neuronal model network that consists of a pair of inhibitory neurons that undergo a frequency dependent synaptic depression.
When the strength of synaptic inhibition between the neurons is varied, such a simple network can display a range of different $n:n$ burst patterns.
Using the timescale disparity between neuronal and synaptic dynamics, we have reduced the network model of eight ODEs to a scalar first return map $\Pi_n$ of the slow depression variable $d$.
This map $\Pi_n$ is a composition of two maps, $F_n$ and $Q_n$, that model the evolution of the depression during the active and silent phases of $n:n$ solutions respectively.
Both $F_n$ and $Q_n$ maps are constructed by using the dynamics of a single uncoupled neuron.
Fixed points of $\Pi_n$ are created in pairs through a fold bifurcation of maps, where the stable fixed point correspond to stable $n:n$ burst solutions of the full two-cell system of ODEs.
The results from our one-dimensional map match excellently with numerical simulation of the full network.
Our results are also in line with \citeauthor{brown1911}'s~\citeyear{brown1911} rhythmogenesis hypothesis, namely that synaptic depression of inhibition is a mechanism by which anti-phase bursting may arise.

% FURTHER METHOD IDEAS
We have studied $n:n$ solutions assuming that the synaptic coupling $\gbar $ between the two cells is symmetrical.
However,~\citet{bose2011} have shown that asymmetrical coupling $(\gbar_1, \gbar_2)$ can result in network solutions of type $m-n$, where one cell fires $m$ spikes, while the other $n$ spikes.
It is conceivable that our map construction can be extended to also capture such $m-n$ solutions.
Remember, in the case of symmetrical coupling with $n:n$ solutions, the timecourse of the depression variables $d_{1}$ and $d_{2}$ were in anti-phase, and it was therefore sufficient to track only one of the two variables.
To capture the full network dynamics in case of asymmetrical coupling one would also have to account for burst patterns of type $m-n$, where the solutions of the depression variables $d_1$ and $d_2$ are not simply time-shifted versions of each other.
To do that, one could track the state of both variables by constructing a two-dimensional Poincaré map $\Pi(d_1,d_2)$.
While geometrical interpretation of two-dimensional maps remains challenging, there exist a number of recent studies which have employed novel geometrical analysis methods to understand the dynamics of two-dimensional maps of small neuronal networks~\citep{akcay2014,akcay2018,liao2020}.
Generally speaking, our map construction approach is applicable to any small network, even with more than two neurons.
As long as the network dynamics occur on separable timescales the main challenges to the map construction lie in identifying the slowest variables, and finding an appropriate, simplified description of their respective timecourses.
In theory, the reduction approach can be also applied to neuronal systems with more than two timescales~\citep[e.g. see][]{kuehn2015}.

In tadpoles, struggling is believed to be initiated by an increase in the firing frequency of reciprocally inhibitory commisural interneurons, which has been hypothesised to lead to stronger synaptic depression of inhibition and result in the iconic anti-phase bursting \citep{li2007}.
It would therefore be interesting to study how varying the cell intrinsic firing period $T$ could affect the network rhythm.
While we have laid out the framework to perform such an investigation, due to the choice of neural model we have avoided varying $T$.
Recall that $T$ is a derived parameter in the~\citet{morris1981} model, and can therefore not be varied in isolation of other model parameters.
This makes verifying any analytical results from our map analysis via numerical integration of the ODEs difficult.
A more abstract model such as the quadratic integrate-and-fire model \citep{izhikevich2004} allows varying $T$ independently of other model parameters, and could be more fitting for such an investigation.

% LIMITATION: SPIKE MAP
Our simulations of the network showed that $n:n$ solutions lose robustness as their period is increased.
That is, solutions with a larger cycle period occur on increasingly smaller intervals of the coupling strength.
We were able to replicate this finding by numerically finding the respective left and right borders of stable $n:n$ branches of fixed points of $\Pi_n$, and showing that the distance between these borders shrinks with $n$.
\linelabel{line:disc-pincA}
We have also noted the resemblance of our bifurcation diagram to one where such $n:n$ branches are created via the bifurcation scenario of type period-increment with co-existent attractors, first described for scalar linear maps with a discontinuity \citep{avrutin2012,avrutin2011}.
It is worthwhile noting that the bifurcations of piecewise linear maps studied by~\citeauthor{avrutin2012} result from a ``reinjection'' mechanism \cite{perez1985}.
Here the orbit of a map performs multiple iterations on one side of the discontinuity, before jumping to the other side and being \textit{reinjected} back into the initial side of the discontinuity.
The stark difference of such a map to our map is that reinjection allows a \emph{single} scalar map to produce periodic solutions of varying periods.
In contrast, we rely on $n$ different maps $\Pi_n$ to describe the burst dynamics without explicitly capturing the period increment dynamics.
It is therefore conceivable that despite the complexity and non-linearity of the dynamics of our two-cell network, a single piecewise-linear map might be already sufficient to capture the mechanisms that shape the parameter space of the full system.
\linelabel{line:disc-pincB}
In their discussion,~\citet{bose2011} briefly outline ideas about how such a linear map could be constructed.

% FLOW SYSTEM BIFURCATION
In addition to stable $n:n$ solutions, the numerical continuation by~\citet{bose2011} also revealed branches of unstable $n:n$ solutions. While we have identified fold bifurcations of our burst map, we have not found corresponding bifurcations of the flow ODE system, and have generally ignored the significance of unstable map fixed points. However, the quadratic nature of the period bifurcation curve is reminiscent of a saddle-node on an invariant circle (SNIC) bifurcation, where the oscillation period lengthens and finally becomes infinite as a limit cycle coalesces with a saddle point.
SNIC bifurcations have been studied in great detail~\cite[e.g.][]{ermentrout1986}, and a next step would be to provide a rigorous explanation of not only the map dynamics, but also of the flow dynamics of the ODE system.
% Should be possible to use averaging to reduce the system.

% BIOLOGICAL IMPORTANCE OF DEPRESSION
% TODO: Perhaps mention learning after all?
We have shown that when the strength of the maximum synaptic conductance is varied, synaptic depression of inhibition can enable our two-cell network to produce burst solutions of different periods.
This result is in line with the idea that one role of synaptic depression in the nervous system may be to allow a finite size neuronal network to participate in different tasks by producing a large number of rhythms~\citep{bose2011, jalil2004, li2007}.
To change from one rhythm to another would only require a reconfiguration of the network through changes in synaptic coupling strength.
Thus short-term synaptic depression of inhibition may provide means for a network to adapt to environmental challenges without changing its topology, that is without the introduction or removal of neurons.
