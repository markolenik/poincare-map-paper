\documentclass[../manuscript.tex]{subfiles}
\graphicspath{{../figures/}}

\begin{document}

\subsection{Anti-phase burst solutions}
Short-term synaptic depression of inhibition in a half-centre oscillator acts as a \emph{burst termination} mechanism~\citep{brown1911} and is known to produce $n-n$ anti-phase burst solutions of varying period.
Such $n-n$ solutions consist of cells firing bursts of $n$ spikes in alternation.
\Cref{fig:depression-traces}A shows the timecourse of a typical $4-4$ burst.
While one cell is firing a burst it provides an inhibitory conductance to the other cell,  preventing it from firing.
Therefore, at any given moment one cell is spiking while the other is inhibited.
Consistent with~\citet{bose2011} we will refer to the currently firing cell as ``free'' and we will call the inhibited cell ``quiet''.
Additionally, we will distinguish between two phases of a $n-n$ solution:
We will refer to the burst duration of a cell as the ``free phase'', which is the time between the first spike and the last spike in a burst. And we will call the remaining duration of a cycle, when a cell is not spiking, the ``quiet phase''.

% TODO: Add mention of gstar. Once critical conductance is crossed, cell is released.
With each action potential of the free cell, short-term depression leads to a step-wise decrease of $d$, and consequently of $s$ (\cref{fig:depression-traces}B).
If $d$ depresses faster at spike time than it can recover in the inter-spike-intervals ($ISI$s), the total synaptic conductance $\gbar s$ will eventually become sufficiently small to allow for the quiet cell to be released and start firing, thus inhibiting the previously free cell.
While a cell is quiet its depression variable can recover.
Once the quiet cell becomes free again its synaptic inhibition will be sufficient to terminate the burst of the previously free cell and commence a new cycle.
As previously demonstrated by~\citet{bose2011}, in a two-cell reciprocally inhibitory network with synaptic depression the coupling strength $\gbar$ determines the type of $n-n$ solution.
Increasing $\gbar$ produces higher $n-n$ burst solutions with more spikes per burst and a longer cycle period.
\Cref{fig:burst-sols} shows numerically stable $n-n$ solutions for varying values of $\gbar$.
For small values of $\gbar$ the network produces anti-phase spiking $1-1$ solutions.
As $\gbar$ is increased the network generates solutions of increasing $n$, that is $2-2$, $3-3$, and $4-4$.
When $\gbar$ is sufficiently large (bottom of \cref{fig:burst-sols}), one of the cells continuously spikes at its uncoupled period $T$ while the other cell remains fully suppressed.
Depending on the initial conditions either of the two cells can become the suppressed cell, which is why the suppressed solution is numerically bistable.

Branches of numerically stable $n-n$ solutions and their associated limit cycle period for varying values of $\gbar$ are depicted in \cref{fig:bif-diagram}A (see  \textbf{Supplementary Material S2} for algorithm description).
Not only do higher $n-n$ solutions branches require stronger coupling $\gbar$, but also within $n-n$ branches the period increases with $\gbar$.
In line with~\citet{bose2011} we find small overlaps between solution branches indicating numerical bistability, for example such as between the $2-2$ and $3-3$ solution branches.
Branches of higher $n-n$ burst solutions occur on increasingly smaller intervals of $\gbar$, for instance is the $\gbar$ interval of the $5-5$ branch shorter than that of the $4-4$ branch and so on.
The interval between the $5-5$ branch and the suppressed solution (region between dotted lines in \cref{fig:bif-diagram}A) not only contains even higher numerically stable $n-n$ solutions, such as $11-11$ bursts, but also other non-symmetric $n-m$ solutions as well irregular, non-periodic solutions. However, the analysis in the following sections will only be concerned with the numerically stable and symmetric $n-n$ solutions.

\subsection{Mathematical analysis of two-cell network}
~\label{sec:assumptions}
% Summary of assumptions
The goal of the following mathematical analysis is to reduce the complexity of the eight-dimensional system to some easily tractable quantity.
As we will see later this quantity is the value of the depression variable $d$ of either of the two cells.
We will construct the solution of $d$ in a piecewise manner from one spike to the next, first during the free phase, and then during the quiet phase.
This construction will require two assumptions about the membrane and synaptic dynamics.
The first assumption states that during a burst the free cell fires at its uncoupled period $T$, which simplifies the construction of the solution of $d$.
The second assumption states that once the inhibitory conductance acting on the quiet cell drops below a critical threshold, the cell is immediately released and fires.
The second assumption is necessary to predict the release time of the quiet cell, which allows us to model the recovery of $d$ during the quiet phase.
In other words, the second assumption requires that the release of the quiet cell from inhibition depends only on the timecourse of the inhibition, and not on the membrane dynamics of the quiet cell.
Both assumptions can be observed in coupled relaxation-oscillator types of neurons such as the Morris-Lecar model we use, and will be numerically verified below.
Both assumptions were first explored in~\cite{bose2011} to derive algebraic conditions that guarantee the periodicity of the depression variable for different $n-n$ solutions.
However here we will use these assumptions to construct a Poincaré map of $d$, which will provide a geometric intuition for the dynamics of the full two-cell network and its dependence on model parameters.

% Assumption 1: Free phase
Our first assumption about the model states that the free cell fires at its uncoupled period $T$, that is, during the free phase of a burst we have $ISI=T$.
Solution profiles in \cref{fig:burst-sols} suggest that the $ISI$s are indeed approximately constant.
We can further numerically confirm this observation by capturing the $ISI$s of the stable solutions from the bifurcation diagram in \cref{fig:bif-diagram}A.
In addition to $ISI$s, \Cref{fig:bif-diagram}B also shows inter-\textit{burst} intervals ($IBI$s), which correspond to the time interval between the last spike of the burst and the full cycle period, and which lie in the quiet phase of the burst.
$IBI$s lie on multiple branches, each branch associated with a stable $n-n$ solution, and are monotonically increasing with $\gbar$.
In contrast, $ISI$s are calculated from the spikes within the free phase and do not vary significantly with $\gbar$, but are approximately $ISI\approx T$, affirming our first assumption.
Assuming $ISI=T$ allows us to ignore the non-linear membrane dynamics during the free phase, and to construct the evolution of the synaptic variables iteratively from spike to spike.
Assuming $ISI=T$ seems reasonable given that inhibition acting on the quiet cell decays exponentially to zero on a much shorter timescale than the duration of the $ISI$, and therefore, once the quiet cell is released its trajectory quickly approaches the spiking limit cycle.

% Assumption 2: Release
Our second assumption states that the quiet cell is released and spikes as soon as inhibition from the free cell drops below a constant threshold.
We will now define such a ``release condition'' by exploiting the discrepancy in timescales between the fast membrane dynamics, and the slower synaptic dynamics.
Let us first consider the dynamics of a single Morris-Lecar neuron.
We fix the synaptic variable that acts on the cell by setting $s=1$, which also makes the applied synaptic conductance $\gbar s$ constant.
Recall from \cref{fig:nullclines}A that in case of a single uncoupled cell ($\gbar s=0$), the $v$- and $w$-nullclines intersect at some unstable fixed point $p_{f}=(v_{f},w_{f})$, while trajectories revolve around a stable spiking limit cycle.
Increasing $\gbar s$ moves the cubic $v_{\infty}$ with the ensuing unstable fixed point $p_{f}$ down the sigmoid $w_{\infty}$ in the $(v-w)$-plane (\cref{fig:nullclines-coupled}).
When $\gbar s$ is large enough, the fixed point $p_{f}$ becomes stable, attracting all previously periodic trajectories.
There exists a unique value $\gbar s=g^{\star}$ when $p_{f}$ changes stability and the stable limit cycle vanishes.
Thus, when a constant inhibitory conductance is applied, $\gbar s<g^{\star}$ acts as a necessary condition for a cell to spike.
In contrast, when $\gbar s>g^{\star}$ inhibition is strong enough to prevent a cell from spiking~\citep{bose2011}.

% NOTE
% In our parameter range the bifurcations occur via subcritical Hopf (bose2011 p.47)

Now let us analyse the nullclines of the quiet cell when the two cells are coupled via synaptic inhibition with depression.
Let $\gbar s$ here denote the total synaptic conductance which acts on the quiet cell and is produced by the free cell, and let $p_{f}$ be the fixed point associated with the quiet cell.
At the start of the burst of the free cell we have $\gbar s > g^{\star}$ and $p_{f}$ is stable.
When the free cell spikes, $\gbar s$ peaks (\cref{eq:s-reset}), and the $v$-nullcline with the ensuing stable $p_f$ move down the $w$-nullcline.
Then in between spikes $\gbar s$ decays exponentially (\cref{eq:dot-s}) causing the $v$-nullcline and $p_{f}$ to move up the $w$-nullcline while attracting trajectories of the quiet cell.
Once depression causes the synaptic conductance to become small enough to satisfy $\gbar s<g^{\star}$ and the quiet cell is released, fixed point $p_{f}$ becomes unstable allowing the quiet cell to fire.
If the trajectory of the quiet cell remains sufficiently close to the stable $p_{f}$ at the time when it changes stability, then $\gbar s < g^{\star}$ acts as a release condition that is not only necessary, but also sufficient for firing of the quiet cell.
In this case the release of the quiet cell occurs precisely when
\begin{equation}
  ~\label{eq:release}
  \gbar s=g^{\star}
\end{equation}
is satisfied.

Whether the $(v,w)$-trajectory of the quiet cell can remain close enough to $p_{f}$ to make \cref{eq:release} sufficient for firing depends largely on the coupling strength $\gbar$ and the timescale disparity between membrane dynamics and synaptic dynamics~\citep{bose2011}.
It is straightforward to test our assumption of a release condition by numerically integrating the full system of ODEs and calculating the time interval between the first spike of the quiet cell and the time when $\gbar s$ first crosses $g^{\star}$.
We will call this time interval the ``release delay''.
If our assumption holds, we would expect an approximately zero release delay.
\Cref{fig:release-delay} shows the numerically computed graph of the release delay for varying $\gbar$.
The graph shows three distinct branches, next to each branch we also plot the timecourse of a corresponding sample solution of the total synaptic conductance $\gbar s$ of both cells.
For the rightmost branch where $\gbar>0.592$ $\si{mS/cm^{2}}$ the release delay is approximately zero.
Here the first spike of the quiet cell can be accurately predicted by the release condition in \cref{eq:release}.
The leftmost and middle branches for $\gbar<0.592$ $\si{mS/cm^{2}}$  show a release delay greater than zero, the quiet cell does not immediately fire when the release condition is first satisfied, and \cref{eq:release} does not accurately predict the release of the quiet cell.
The leftmost and middle branches contain $1-1$ and $2-2$ solutions respectively.
In both cases the coupling $\gbar$ is not sufficiently strong to allow trajectories of the quiet cell to be close enough to $p_{f}$ to guarantee spiking once $\gbar$ crosses $g^{\star}$.
Note that in the middle branch $\gbar s $ crosses $g^{\star}$ twice, and only after the second crossing does the quiet cell fire.
The following map construction relies on the assumption that the release condition in \cref{eq:release} can accurately predict the release time of the quiet cell.
Given our model parameters this is only possible for sufficiently large $\gbar>0.592 $ $\si{mS/cm^{2}}$.
For completeness, however, we will also consider values $\gbar < 0.592$ $\si{mS/cm^{2}}$ in the following analysis, bearing in mind that in this parameter range our map will not be accurate.

In summary: For $\gbar>0.592$ $\si{mS/cm^{2}}$ the release condition is sufficient to predict when the quiet cell is released.
Due to the symmetry of $n-n$ solutions the release occurs at exactly half the period of the full cycle, that is at $P/2$.
The release time therefore uniquely determines the type of $n-n$ solution.
Furthermore, computation of the release time does not depend on the membrane nor the synaptic dynamics of the quiet cell.
Instead, the solution of the synaptic variable $s$ of the free cell is sufficient to predict when $\gbar s=g^{*}$ is satisfied.
Finally, $s$ is solely determined by the evolution of the depression variable $d$ of the free cell.
Constructing a solution of $d$ during the free phase of either cell will therefore uniquely determine the solution of the full eight-dimensional network.
However, finding the solution $d$ requires us to know the initial value $d(0)$ at the start of a cycle at $t=0$.
In the next section we will construct a scalar return map that tracks these initial values $d(0)$ from cycle to cycle of stable $n-n$ solutions.

\subsection{Construction of the scalar Poincaré map}
In this section we construct the scalar Poincaré map $\Pi_{n}:d^{\star}\mapsto d^{\star}$.
Here the discrete variable $d^{\star}$ tracks the values of the continuous depression variable $d$ at the beginning of each $n-n$ burst.
The map $\Pi_{n}$ therefore describes the evolution of $d$, of either of the two cells, from the beginning of one cycle to the beginning of the next cycle.
To simplify the map construction we will assume that a free cell fires exactly $n$ times before it becomes quiet.
Later we will relax this assumption.
We will construct $\Pi_{n}$ by evolving $d$ first during the free phase and then during the quiet phase of the $n-n$ limit cycle.
First, let us give explicit definitions of the free and quiet phases.
A schematic illustration of both phases is given in \cref{fig:free-quiet1}.

Suppose that at $t=0$ cell 1 becomes free with some initial $d(0)$.
Cell 1 then fires $n$ spikes at the uncoupled period $T=T_{act}+T_{inact}$.
Let $s(t)$ and $d(t)$ be the corresponding solutions of the synaptic and depression variables of cell 1.
After $n$ spikes the total conductance $\gbar s(t)$ acting on the quiet cell 2 has decayed sufficiently to satisfy the release condition~\eqref{eq:release}, that is at some time $t=(n-1)T + \delt$, where $\Delta t < T_{inact}$, we have $\gbar s(t)=g^{\star}$.
Cell 2 is then released and prevents cell 1 from further spiking.
Here $\delt$ is the time between the last spike of cell 1 and the first spike of cell 2 ~\citep{bose2011}.
Once released, cell 2 also fires $n$ spikes until cell 1 becomes free once again at the cycle period.
Let $P_n$ denote the full cycle period of a $n-n$ solution:
\begin{equation}
  ~\label{eq:P}
  P_n = 2(n-1)T + 2\delt.
\end{equation}
We can now define the free and quiet phases of cell 1 explicitly. The free phase is the time interval between the first and last spikes of the burst, that is for time $0<t<(n-1)T$.
During the free phase of cell 1, the quiet cell 2 is inhibited sufficiently strong to prevent it from firing, hence $\gbar s > g^{\star}$.
The quiet phase of cell 1 is the remaining duration of the cycle when the cell is not firing, that is for $(n-1)T < t < 2(n-1)T + 2\delt$.

Note that only the quiet phase depends on $\delt$ which will play a central role in the construction of $\Pi_{n}$.
From \cref{eq:P} $\delt$ can be be computed as
\begin{equation}
  ~\label{eq:delta-t-P}
  \delt = \frac{1}{2}P_n - (n-1)T.
\end{equation}
\noindent
We can use \cref{eq:delta-t-P} and the numerically computed bifurcation diagram of the period for stable $n-n$ solutions in \cref{fig:bif-diagram}A to obtain the graph of $\delt$ as a function of $\gbar$ (\cref{fig:delta-t}).
Each continuous branch of $\delt$ is monotonically increasing and corresponds to a $n-n$ burst:
Stronger coupling $\gbar$ increases the total synaptic conductance $\gbar s$ that acts on the quiet cell, thus delaying its release.
It is easy to see that for any $n$-branch we have $\delt<T$:
Once $\delt$ crosses $T$, the free cell can ``squeeze in" an additional spike and the solutions bifurcate into a $(n+1)-(n+1)$ burst.

Distinguishing between the active and silent phases of a cycle allows us to describe the dynamics of the depression variable $d$ explicitly for each phase.
As can be seen from \cref{fig:free-quiet1}C, during the active phase $d$ depresses during the active phase of spikes and recovers during the inactive phases of spikes.
% TODO need to sort out terminology here, i.e. inactive vs quiet
In contrast, during the silent phase $d$ only recovers and does not depress.
Given the initial $d^{\star}=d(0)$ at the beginning of the cycle and the number of spikes in the free phase $n$, we can now construct the burst map $\Pi_{n}$.
The map
\begin{equation}
  \Pi_{n}(d^{\star})=Q_{n}\big(F_{n}(d^{\star}\big))
\end{equation}
\noindent
is a composition of two maps.
Map
\begin{equation}
  F_{n}:d^{\star}\mapsto \delt
\end{equation}
models the evolution of $d$ in the free phase.
$F_{n}$ takes an initial value $d^{\star}$ and calculates the inter-burst-interval $\delt$.
Map
\begin{equation}
  Q_{n}:\delt \mapsto d^{\star}
\end{equation}
models the recovery of $d$ in the quiet phase.
Given some $\delt$ map $Q_n$ computes $\dstar$ at the start of the next cycle.

Our aim in the following analysis is to elucidate the properties of $\Pi_{n}$ and to understand the structure of its parameter space by exploring how the stable and unstable fixed points of $\Pi_{n}$ are created.
To that effect it will be useful to include not only positive, but also negative values of $d^{\star}$ to the domain of $\Pi_{n}$.
But it is important to add that values $d^{\star}<0$ are biologically impossible as the depression variable models a finite pool of neurotransmitters, and therefore must be positive.
Because $\Pi_{n}$ maps first from $d^{\star}$ to $\delt$, and then back to $d^{\star}$, we will also consider negative values of $\Delta t$, interpreting them as $n-n$ solutions with partially overlapping bursts.
As will become evident, $\delt<0$ is only a formal violation of the biological realism of the map $\Pi_{n}$, as numerically stable $n-n$ solutions of the full system of ODEs only exist for $\Delta t>0$.

% TODO: Edit par
We start the construction of $\Pi_n$ by first considering the free phase and building the map $F_n$.
At each spike time $t_{k}$ where $d(t_k) = d_k$, variable $d$ decays first for the duration of the active phase of the spike for $T_{act}$, as described by the solution to \cref{eq:dot-d}.
At $t = t_k + T_{act}$ we have
\begin{equation}
  d(t_k + T_{act}) = d_k e^{-T_{act}/\tau_b}.
\end{equation}
The depression variable then recovers during the inactive phase of the spike until $t_{k+1}$, where for $0<t<T_{inact}$ we get
\begin{equation}
  d(t_{k+1}) = 1 - (1 - d_k e^{-T_{act}/\tau_b} )e^{-t/\tau_a}.
\end{equation}
By substituting $t=T_{inact}$ we can build a linear map that models the depression of $d$ from spike time $t_{k}$ to the subsequent spike time $t_{k+1}$ during the free phase:
\begin{equation}
  d_{k+1} = \lambda\rho d_{k} + (1-\rho),~\label{eq:map-d}
\end{equation}
where to keep the notation simple we let
\begin{align}
  \lambda & = \exp(-T_{act}/\tau_b),    \\
  \rho    & =\exp(-T_{inact}/\tau_{a}).
\end{align}

Given constant $T_{act}$ and $T_{inact}$, parameter $\lambda$ determines how much the synapses depresses during the active phase of the spike, while $\rho$ determines how much it recovers during the inactive phase.
Since $0<\lambda, \rho<1$, map \cref{eq:map-d} is increasing and contracting, with a fixed point at
\begin{equation}
  ~\label{eq:dsup}
  d_{s}=\frac{1-\rho}{1-\lambda\rho},
\end{equation}
where $0<d_{s}<1$.
The value $d_{s}$ is the maximum depression value that can be observed in the suppressed solution where the active cell fires at its uncoupled period $T$ (see \cref{fig:burst-sols}E).
Using the release condition in \cref{eq:release} allows us to derive the value of the the minimum coupling strength that will produce the full suppressed solution, denoted as $\gbar_{s}$.
% TODO Explain why s(0) = ds
% TODO See https://trello.com/c/AYqZIaXh: Would be good to elaborate on the idea that lambda and tau can be
% viewed as the new time constants in the reduced system.
Solving \cref{eq:dot-s} for $s(t)$ with $t=T$ and setting the initial value $s(0)=d_{s}$ gives
\begin{equation}
  \gbar_{s} d_{s}e^{-T/\tau_\kappa}=g^{\star}.
\end{equation}
By further substituting the definition of $d_{s}$ in~\eqref{eq:dsup} and rearranging, we can write $\gbar_{s}$ as a function of $\lambda$ and $\rho$:
\begin{equation}
  \label{eq:gs}
  \gbar_{s}(\lambda, \rho) = g^{\star}e^{T/\tau_\kappa}\frac{1-\lambda \rho}{1-\rho}.
\end{equation}
Note that the above dependence of $\gbar_{s}$ on $\lambda$ is linear and monotonically decreasing.
Increasing $\lambda$ reduces the strength of the depression of the free cell.
This in turn allows the free cell to fully suppress the quiet cell at smaller values of $\gbar$.

Solving~\eqref{eq:map-d} gives us the linear map $\delta_{n}:d^{\star}\mapsto
d_{n}$, that for some initial $d^{\star}$ computes the depression at the $n$th
spike time, $d_{n}=d(t_{n}^{-})$:
\begin{equation}
  ~\label{eq:delta-map}
  \delta_{n}(d^{\star}) = (\lambda \rho)^{n-1} d^{\star} + (1-\rho)\sum_{i=0}^{n-2}(\lambda \rho)^{i}.
\end{equation}
Since $\lambda < 1$, function $\delta_{n}$ is a linearly increasing function of
$d^{\star}$ with a fixed point at $d_{s}$ for all $n$.  Having identified $d$ after $n$
spikes, we can now use the release condition $\gbar s = g^{\star}$ (\cref{eq:release}) to
find $\delt$.
% FIXME:
After the last spike of the free phase at time $t_{n} = (n-1)T$ the synapse variable $s$
has the value of $d$ for the duration of $T_{act}$, which is given by
$\delta_n(\dstar)\lambda$. $s$ then decays exponentially for $\Delta t < T_{inact}$.
Solving \cref{eq:dot-s} (case $v < v_\theta$) with initial condition
$s(0)=\dn(\dstar)\lambda$ yields:
\begin{equation}
  ~\label{eq:s-sol}
  s(\delt)=\dn(\dstar)e^{-\Delta t/\tau_\kappa}.
\end{equation}
Substituting $s(\delt)$ into $s$ of the release condition (\cref{eq:release})
gives then
\begin{equation}
  ~\label{eq:release2}
  \gbar \dn(\dstar) e^{-\delt/\tau_\kappa}=g^{\star}.
\end{equation}
Our assumption of the release condition guarantees that the quiet cell 2 spikes and
becomes free when $\gbar s - g^{\star}$ crosses zero. Solving \cref{eq:release2} for
$\delt$ allows us to compute the inter-spike-interval as a function of $\dstar$, which
defines our map $F_n$:
\begin{equation}
  ~\label{eq:Fn-map}
  F_{n}(d^{\star}):=\tau_\kappa \ln{\left(\frac{\gbar }{g^{\star}} \delta_{n}(d^{\star})\right)}= \delt.
\end{equation}

\Cref{fig:FQ-map}A shows $F_n$ for various $n$, which is a strict monotonically increasing
function of $d^{\star}$ as well as $\gbar$.  Larger values of $d^{\star}$ and $\gbar$,
respectively, cause stronger inhibition of the quiet cell, and therefore prolong its
release time and the associated $\delt$.  Map $F_{n}$ is defined on $d^{\star}>d_{a}$,
where $d_{a}$ is a vertical asymptote found by solving $\delta_{n}(\dstar)=0$ in
\cref{eq:delta-map} for $\dstar$, which yields
\begin{equation}
  d_{a}(n)=-\frac{(1-\rho)\sum_{i=0}^{n-2}(\lambda \rho)^{i}}{ (\lambda \rho)^{n-1} }\leq 0~\label{eq:da}.
\end{equation}

We now turn to the construction of map $Q_{n}$, which describes the recovery of the depression variable during the quiet phase.
As we have identified earlier, the recovery in the quiet phase of a $n-n$ solution is of duration $2\delt +(n-1)T$.
Substituting that into the solution for $d(t)$ (\cref{eq:d-sol}) with the initial condition $d(0)=\dn(\dstar)$ yields the map $Q_{n}$:
\begin{equation}
  ~\label{eq:Qn-map}
  Q_{n}(\delt):=1- (1- \lambda \dn(\dstar))e^{-(2\Delta t+(n-1)T)/\tau_{d}}.
\end{equation}
Given $\delt$, we can find $\dn(\dstar)$ by rearranging the release condition in \cref{eq:release2}:
\begin{equation}
  ~\label{eq:dn}
  \dn(\dstar) = \frac{g^{\star}}{\gbar} e^{\delt/\tau_{s}}.
\end{equation}
Map $Q_{n}$ is shown in \cref{fig:FQ-map}B for various values $n$.
Note that $Q_{n}$ is monotonically increasing as larger values $\delt$ imply a longer recovery time, and hence $Q_{n}$ grows without bound.
All curves $Q_{n}$ intersect at some $\delt = \tau_{s}\ln{\left[\gbar/(g^{\star}\lambda)\right]}$ where
\begin{equation}
  ~\label{eq:Qn-intersect}
  Q_{n}\left[\tau_{s}\ln{\left(\frac{\gbar}{g^{\star}\lambda}\right)}\right]=1.
\end{equation}
As we will show in the next section, all fixed points of the full map $\Pi_{n}$ occur for $d^{\star}<1$.
We will therefore restrict the domain of $Q_{n}$ to $(-\infty, \ln{\left[\gbar/(g^{\star}\lambda)\right]}\tau_{s})$ and the codomain to $(-\infty, 1)$.
Additionally, while values $\delt>T$ will be helpful in exploring the geometry of $\Pi_{n}$, recall from \cref{fig:delta-t} that in the flow system all $n-n$ solutions bifurcate into $(n+1)-(n+1)$ solutions exactly when $\Delta t=T$, and we will address this concern in the last part of our map analysis.

Having found $F_{n}$ and $Q_{n}$, we can now construct the full map $\Pi_{n}(d^{\star})=Q_{n}\big(F_{n}(d^{\star})\big)$:
\begin{equation}
  ~\label{eq:Pn-map}
  \Pi_{n}(d^{\star})= 1-
  \frac{\rho^{n-1}{g^{\star}}^{\tau}}{{\gbar}^{\tau}}\delta_{n}^{-\tau}(d^{\star})\big(1-\lambda\delta_{n}(d^{\star})\big),
\end{equation}
where we substituted $\tau = 2\tau_{s}/\tau_{d}$.
Since $d$ is the slowest variable of the system and $\tau_{d}\gg \tau_{s}$, we will also assume $\tau<1$.
\Cref{fig:Pn-map}A depicts $\Pi_{n}$ for various $n$.
Intersections of $\Pi_{n}$ with the diagonal are fixed points of the map.
\Cref{fig:Pn-map}B shows $\Pi_{2}$ with $n=2$.
Varying the synaptic strength $\gbar$ moves the curves $\Pi_{n}$ up and down the $(d^{\star}, \Pi_{n})$-plane.
For $\gbar<0.03$ $\si{mS/cm^{2}}$ map $\Pi_{2}$ has no fixed points.
As $\gbar$ is increased to $\gbar \approx 0.03$ $\si{mS/cm^{2}}$, curve $\Pi_{2}$ coalesces with the diagonal tangentially.
When $\gbar > 0.03$ $\si{mS/cm^{2}}$, a pair of fixed points emerge, one stable and one unstable fixed point, indicating the occurrence of a fold bifurcation of maps.


From \cref{eq:Pn-map} it is evident that $\Pi_{n}$ is monotonically increasing with respect to $\gbar$ and also $\dstar$:
\begin{align}
  \label{eq:non-degen1}
  \dv{\Pi_{n}}{\gbar}  & >0, \\[0.5ex]
  \label{eq:dPidd}
  \dv{\Pi_{n}}{\dstar} & >0,
\end{align}
and in the following sections we will heavily rely on this monotonicity property of $\Pi_n$.
Just as $F_{n}$, curves $\Pi_{n}$ spawn at the asymptote $d_{a}$ (\cref{eq:da}), and because
\begin{equation}
  \lim_{\gbar \to \infty}\Pi_{n} = 1\text{ for all }n,
\end{equation}
fixed points of $\Pi_{n}$ lie in $(d_{a}, 1)$.

% \subsection{Existence and stability of fixed points of $\mathit{\Pi_{n}}$}
\subsection{Existence and stability of fixed points}
We introduce the fixed point notation $\dstar_{f}$ with $\Pi_{n}(\dstar_{f})=\dstar_{f}$.
The existence of fixed points $\dstar_{f}$ for $\gbar$ sufficiently large can be shown from the strict monotonicity of $\Pi_{n}$ with respect to $\gbar$ and $\dstar$ (\cref{eq:dPidd,eq:non-degen1}), as well as the fact that the slope of $\Pi_{n}$ is monotonically decreasing,
\begin{equation}
  \label{eq:non-degen2}
  \left(\frac{\mathrm{d}}{\mathrm{d}\dstar}\right)^2 \Pi_{n}<0.
\end{equation}
In the limit $\dstar \to d_{a}$ the value of $\Pi_n$ decreases without bound for any $\gbar>0$.
In the limit $\gbar \to 0$, $\Pi_n$ also decreases without bound, but as $\gbar \to \infty$ values of $\Pi_n$ approach $1$.
It follows from \cref{eq:non-degen1} and the intermediate value theorem that for some $\gbar$ large enough $\Pi_n$ intersects the diagonal.
Moreover, because $\Pi_n$ and its slope are monotonic with respect to $\dstar$, there exists some critical fixed point $(\dstar_b, \gbar_b)$ where $\Pi_n$ aligns with the diagonal tangentially with
\begin{align}
  \Pi_{n}(\dstar_{b}; \gbar_{b})                     & =\dstar_{b}, \\
  \frac{\d}{\d \dstar}\Pi_{n}(\dstar_{b}; \gbar_{b}) & =1.
\end{align}
Equations~\eqref{eq:non-degen1} and~\eqref{eq:non-degen2} constitute the non-degeneracy conditions for a codimension-1 fold bifurcation of maps, indicating that in a neighbourhood of $(\dstar_{b}, \gbar_{b})$ map $\Pi_{n}$ has the topological normal form described by the graph of
\begin{equation}
  \label{eq:normal-form}
  x\mapsto \beta+x-x^{2},
\end{equation}
with a stable and unstable fixed point $x=\pm\sqrt{\beta}$, and slopes $\dv*{x}{\beta}=\mp {(2\sqrt{\beta})}^{-1}$, respectively.

\subsection{Fold bifurcations}
Fixed points of $\Pi_n$ satisfy the fixed point equation
\begin{equation}
  \label{eq:Phi}
  \Phi_n(\dstar; \gbar)=0,
\end{equation}
where
\begin{equation}
  ~\label{eq:Phi-def}
  \Phi_{n}(d^{\star}, \gbar):=\Pi_{n}(d^{\star}, \gbar)-d^{\star}.
\end{equation}
As we have already shown, for $\gbar > \gbar_b(n)$ solutions to \cref{eq:Phi} exist in pairs of stable and unstable fixed points.
Solving \cref{eq:Phi} explicitly for $\dstar$ it not trivial, but solving for $\gbar$ is straightforward and given by $\gbar = G_n(\dstar)$, where
\begin{equation}
  ~\label{eq:g}
  G_{n}(d^{\star}) := g^{\star} \left(\frac{\rho^{n-1}\delta_{n}^{-\tau}(d^{\star})(1-\lambda\delta_{n}(d^{\star}))}{1-d^{\star}}\right)^{1/\tau}.
\end{equation}
Plotting $d^{\star}$ against $\gbar$ gives the fixed point curves, which are shown in \cref{fig:folds}A.
Note the typical quadratic shape of a fold bifurcation of maps.
It is also evident that the fold bifurcations occur for increasingly smaller $\gbar$ as $n$ is increased.
Moreover, we can observe that unstable fixed points have negative values of $d^{\star}$ for $n>1$.

\Cref{eq:g} also allows us to find the critical fixed point connected with the fold bifurcation, namely $\big(\dstar_{b}(n), \gbar_{b}(n)\big)$, which is the global minimum of $G_{n}(\dstar_f)$:
\begin{align}
  \dstar_{b}(n) & = \operatorname{argmin} G_{n}(\dstar_{f}), \\
  \gbar_{b}(n)  & = \min{G_{n}(\dstar_{f})}.
\end{align}
Function $G_{n}$ is strictly monotonic on the respective intervals of $\dstar_f$ that correspond to the stable and unstable fixed points, that is
\begin{align}
  \dv{G_n}{\dstar_f} & < 0, \text{ for } \dstar_f>\dstar_b(n) \text{ stable},   \\[0.5ex]
  \dv{G_n}{\dstar_f} & > 0, \text{ for } \dstar_f<\dstar_b(n) \text{ unstable},
\end{align}
which allows us to express the stable and unstable fixed points as the inverse of $G_n$ on their respective intervals of $\dstar_f$.
Because we are primarily interested in the stable fixed points, we define the stable fixed point function $\dstar_f=\phi_n(\gbar)$ as
\begin{equation}
  \label{eq:phi}
  \phi_{n}(\gbar):= G_{n}^{-1}(\dstar_{f}) \text{ for } \dstar_{f}>\dstar_{b}(n).
\end{equation}
Function $\phi_n(\gbar)$ is also monotonic, and is therefore straightforward to compute numerically via root-finding.
Here we use the Python package Pynverse~\citep{pynverse} for that purpose.

Having found the stable fixed points $\dstar_f$ as a function of the coupling strength $\gbar$, we can now compute the associated cycle period.
Recall that the period is given by \cref{eq:P}, which we can be written as a function of $\gbar$:
\begin{equation}
  \label{eq:period}
  P_n(\gbar) = 2(n-1)T + 2F_n\big(\underbrace{\phi_n(\gbar)}_{\dstar_f}, \gbar\big),
\end{equation}
where map $F_n$ (\cref{eq:Fn-map}) calculates the inter-burst-interval $\delt$ given a stable fixed point $\dstar_f=\phi_n(\gbar)$.
We plot the predicted period $P_n(\gbar)$ versus the cycle period that was computed from numerically integrating the full system of ODEs in \cref{fig:folds}B.
For $n>1$ our map $\Pi_n$ accurately predicts the period.
When laying out our assumptions in \cref{sec:assumptions}, we have already predicted an inaccuracy for $n=1$ (see \cref{fig:release-delay}), since here $\gbar$ is not sufficiently strong to guarantee the validity of our release condition (\cref{eq:release}).

It is evident from \cref{fig:folds}A that $\phi_n$ is strictly increasing with $\gbar$.
This property follows directly from the normal form of the fold bifurcation (\cref{eq:normal-form}), but can also be shown using implicit differentiation and the fixed point equation $\Phi_n(\phi_n(\gbar), \gbar)=0$ in \cref{eq:Phi}. For $\dstar_f=\phi_n(\gbar)>d_b(n)$ we get:
\begin{equation}
  \label{eq:dstardg}
  \dv{\phi_n}{\gbar} = -\frac{\pdv*{\Phi_n}{\gbar}}{\pdv*{\Phi_n}{\dstar}} =
  \frac{\pdv*{\Pi_n}{\gbar}}{1-\pdv*{\Pi_n}{\dstar}}>0.
\end{equation}
The inequality follows from $\pdv*{\Pi_n}{\gbar}>0$ and the fact that $\pdv*{\Pi_n}{\dstar}<1$ for $\dstar>d_b(n)$.
\Cref{eq:dstardg} allows us to explain why the period $P_n$ increases with $\gbar$, as seen in \cref{fig:folds}B.
Differentiating $P_n$ gives:
\begin{equation}
  \label{eq:dPdg}
  \dv{P_{n}}{\gbar} = 2\grad F_n(\dstar_f, \gbar) \cdot
  \begin{bmatrix}\pdv*{\phi_n}{\gbar}\\[0.5ex] 1\end{bmatrix}>0,
\end{equation}
where the partial derivatives of $F_n(\dstar_f, \gbar)$ are:
\begin{align}
  \pdv{F_n}{\dstar_f} & = \taus \frac{(\lambda\rho)^{n-1}}{\dn(\dstar_f)}>0, \\[0.5ex]
  \pdv{F_n}{\gbar}    & = \frac{\taus}{\gbar}>0.
\end{align}
\Cref{eq:dstardg,eq:dPdg} have an intuitive biological interpretation:
Increasing the coupling strength between the neurons leads to overall stronger inhibition of the quiet cell, which delays its release and leads to a longer cycle period.
The latter allows more time for the synapse to depress in the free phase and recover in the quiet phase, resulting in overall larger values of $\dstar_f$, that is weaker depression at the burst onset.

While fixed points of our Poincaré map predict the cycle period of the flow system excellently, its construction relies on the strong assumption that the free phase contains exactly $n$ spikes.
As is evident from \cref{fig:folds}B this assumption is clearly violated in the flow system, as stable $n-n$ bursts exists only on certain parameter intervals of $\gbar$.
In the last sub-section we will analyse the mechanisms that guide how the stable $n-n$ are created and destroyed, and use our previous analysis to derive the corresponding parameter intervals of $\gbar$ where such solutions exist.

% TODO: Need to find a better sub-section title that doesn't include 'period increment'
\subsection{Period increment bifurcations with co-existent attractors}
Let $\gbar_+(n)$ and $\gbar_-(n)$ denote the left and right parameter borders on $\gbar$ where stable $n-n$ solutions exist.
That is, as $\gbar$ is increased stable $n-n$ solutions are created at $\gbar_+(n)$ and destroyed at $\gbar_-(n)$.
When $\gbar$ is reduced beyond $\gbar_+(n)$, $n-n$ solutions bifurcate into $(n-1)-(n-1)$ solutions, while when $\gbar$ is increased beyond $\gbar_-(n)$, $n-n$ solutions bifurcate into $(n+1)-(n+1)$ solutions.
Let us briefly recap our observations regarding $\gbar_+(n)$ and $\gbar_-(n)$ from the numerical bifurcation diagram in \cref{fig:folds}B.
For $n>1$ there are the following relations:
\begin{align}
  \gbar_+(n)                  & < \gbar_-(n)\label{eq:easy1},                                       \\
  \gbar_+(n)                  & < \gbar_+(n+1)\text{ and } \gbar_-(n)<\gbar_-(n+1)\label{eq:easy2}, \\
  \gbar_+(n+1)                & < \gbar_-(n)\label{eq:coexistence},                                 \\
  \gbar_-(n+1) - \gbar_+(n+1) & < \gbar_-(n) - \gbar_-(n)\label{eq:robustness}.
\end{align}
\Cref{eq:easy1,eq:easy2} are self-explanatory.
\Cref{eq:coexistence} formally describes occurrence of co-existence between stable $n-n$ and $(n+1)-(n+1)$ solutions.
\Cref{eq:robustness} implies that the parameter interval on $\gbar$ of $n-n$ solutions decreases with $n$, in other words, bursts with more spikes occur on increasingly smaller intervals of the coupling strength.
All of the above relations are reminiscent of the period increment bifurcations with co-existent attractors, first described for piecewise-linear scalar maps with a single discontinuity by Avrutin and colleagues~\cite[e.g. see][]{gardini2012,tramontana2012,avrutin2011,gardini2012}.
While our maps $\Pi_n$ are fully continuous, the above observation suggests that
a different piecewise-linear scalar map that captures the period increment bifurcations of the full system might exist.
We will explore what such a map might look like in the discussion.

Let us now find algebraic equations that will allow us to calculate the critical parameters $\gbar_+(n)$ and $\gbar_-(n)$ associated with the period increment bifurcations.
% BUG: Maybe don't call it period-increment here?
Recall that the period $P_n$ derived from the fixed points of $\Pi_n$ is an increasing function of $\gbar$:
\begin{equation}
  \dv{P_{n}}{\gbar}=2\dv{F_n(\phi_n(\gbar), \gbar)}{\gbar}>0,
\end{equation}
that is, as the coupling strength increases, it takes longer for the total synaptic conductance to fall below the value of the release conductance, which delays the release of the quiet cell, and $\delt$ becomes larger.
Once $\delt=T$, the free cell can produce another spike and the solution bifurcates into a $(n+1)-(n+1)$ solution.
Note, however, that at $\gbar_+(n)$ the bifurcation into a $(n-1)-(n-1)$ does not occur when $\delt=0$.
Here the mechanism is different: A sufficient reduction of $\gbar$ causes the total synaptic conductance to drop below the release conductance in the \emph{previous} $ISI$, which allows the quiet cell to be released one spike earlier.

Using the above reasoning we can now formulate the conditions for both bifurcations at $\gbar_+(n)$ and $\gbar_-(n)$.
As in the previous sections, we will only restrict ourselves to the analysis of the stable fixed points given implicitly by $\dstar_f=\phi_n(\gbar)$ (\cref{eq:phi}).
At the right bifurcation border $\gbar_-(n)$ we have $\delt=T$, and after substituting our $F_{n}$-map (\cref{eq:Fn-map}) this translates into
\begin{equation}
  F_n(\phi_n(\gbar), \gbar) = T,
\end{equation}
which lets us define a function
\begin{equation}
  \label{eq:R}
  R_{n}(\gbar):=F_n(\phi_n(\gbar), \gbar)-T,
\end{equation}
whose root is the desired right bifurcation border $\gbar_-(n)$.
In case of the left bifurcation border at $\gbar_+(n)$ the release condition is satisfied just before the free cell has produced its $n$th spike, and after the depression variable has been reset $n-1$ times, which gives the condition
\begin{equation}
  \gbar \delta_{n-1}\big(\phi_{n}(\gbar)\big)e^{-T/\taus} = \gstar,
\end{equation}
and can be rewritten as a function
\begin{equation}
  \label{eq:L}
  L_{n}(\gbar):=\gbar \delta_{n-1}\big(\phi_{n}(\gbar)\big)e^{-T/\taus} -\gstar,
\end{equation}
whose root is $\gbar_+(n)$.
Both $R_{n}$ and $L_{n}$ are increasing with respect to $\gbar$, which makes finding their roots numerically straightforward.

\Cref{fig:final-bif} shows the period $P_n(\gbar)$ as predicted by the fixed points of $\Pi_n$ (\cref{eq:period}) plotted on their respective intervals $\gbar \in [\gbar_+(n),\gbar_-(n)]$ (blue), as well as the cycle period acquired from numerical integration of the full system of ODEs (orange).
Note that the width of $n-n$ branches decreases with $n$, which confirms the inequality in \cref{eq:robustness}.
That is, bursts with more spikes occur on increasingly smaller intervals of $\gbar$, which can be interpreted as a lost of robustness with respect to the coupling strength of long-cyclic solutions.
We also note the occurrence of bi-stability between pairs of $n-n$ and $(n+1)-(n+1)$ branches, which also confirms our initial observation in \cref{eq:coexistence}.

As previously observed in \cref{fig:folds}B our maps prediction of the cycle period is accurate for $n>1$.
Recall that our reduction assumptions required a sufficiently large coupling strength, which we numerically estimated to be $\gbar \approx 0.592\si{mS/cm^2}$ in \cref{fig:release-delay}.
The mismatch in period for $1-1$ solutions, but also the mismatch in the left bifurcation border $\gbar_-(n=2)$ of the $2-2$ solution can be attributed to the violation of that assumption.
However, even for branches of large $n-n$ solutions there is a mismatch between the bifurcation borders.
Presumably our assumptions on the time scales of $w$ and $s$ dynamics do not hold here, and can only be captured by more complex approximations.
Nevertheless, our map allows approximate extrapolation of the cycle period and the respective bifurcation borders where numerical integration of the ODEs would require a very small time step.

\end{document}